\documentclass[a4paper,12pt]{article}
\usepackage[utf8]{inputenc}
\usepackage[demo]{graphicx}
\usepackage{hyperref}
\usepackage{array}

\usepackage{lastpage}
\usepackage{lipsum}
\usepackage{boldline} 
\usepackage{tabularx} % in the preamble
\usepackage{fancyhdr}
\usepackage[hmargin=2cm,top=4cm,headheight=130pt,footskip=65pt]{geometry}
\renewcommand{\familydefault}{\sfdefault}

\hypersetup{
    colorlinks=true,
    linkcolor=blue,
    filecolor=magenta,      
    urlcolor=cyan,
}

\urlstyle{same}

\setlength{\parindent}{0.95cm}

\pagestyle{fancy}
\renewcommand{\headrulewidth}{0pt}
\fancyhead[CE,CO,LE,LO,RE,RO]{} %% clear out all headers
\fancyhead[C]{%
           \begin{flushleft}
          \includegraphics[height=1.5cm,width=2.5cm]{logo.png}
          \end{flushleft}
          \bgroup
          \def\arraystretch{1.5}% 
          \begin{tabular}{V{4}l|l|l|lV{4}}
            \hlineB{4}
            \multicolumn{4}{V{4}cV{4}}{
                \textbf{\textless{}Document Title\textgreater{}}
            } \\ 
            \hline
            {\footnotesize Type: \textless{}Type\textgreater{}} & 
            {\footnotesize No.: \textless{}Number\textgreater{}} & 
            {\footnotesize Version: \textless{}Document Version\textgreater{}} & 
            {\footnotesize Page: \thepage of \pageref{LastPage}} \\ 
            \hlineB{4}
          \end{tabular}%
          \egroup
}


\begin{document}

\section{Purpose}
This document defines the hardware and software test procedures required for verification of a
self-sealing stem bolt. As we know nothing could be built without bolts. They are a basic component of reverse-ratcheting routing planers.
\section{References}

\subsection{Internal References}

\begin{itemize}
	\item       Self-sealing stem bolt on memory-alpha.fandom.com (\href{https://memory-alpha.fandom.com/wiki/Self-sealing_stem_bolt}{link})
	\item  Self-sealing stem bolt on memory-beta.fandom.com (\href{https://memory-beta.fandom.com/wiki/Self-sealing_stembolt}{link})
\end{itemize}

\section{System Verification}
\subsection{Objective}
\subsection{Required Equipment}
\subsection{Setup and Configuration}
\subsection{Requirements Tested}



\begin{tabularx}{\textwidth}{|X|}
	\hline
	\textbf{Requirements from Document Number:}t \\ Reference the document where the requirements are coming from here \\ \hline
	S3.1.1, S3.2.2, S3.2.3, S3.2.4               \\ \hline
\end{tabularx}


\subsubsection{Procedure and Test Worksheet}



\end{document}
