\documentclass[a4paper,12pt]{article}
\usepackage[utf8]{inputenc}
\usepackage[demo]{graphicx}
\usepackage{hyperref}
\usepackage{array}
\usepackage{lastpage}
\usepackage{lipsum}
\usepackage{boldline} 
\usepackage{tabularx, colortbl}
\usepackage{fancyhdr}
\usepackage{xltabular}
\usepackage{amssymb}
\usepackage[hmargin=2cm,top=4cm,headheight=130pt,footskip=65pt]{geometry}
\renewcommand{\familydefault}{\sfdefault}

\hypersetup{
    colorlinks=true,
    linkcolor=blue,
    filecolor=magenta,      
    urlcolor=cyan,
}
\newcounter{teststepctr}


\definecolor{gray}{cmyk}{0.05,0.05,0.05,0.05}

\urlstyle{same}

\setlength{\parindent}{0.95cm}

\pagestyle{fancy}
\renewcommand{\headrulewidth}{0pt}
\fancyhead[CE,CO,LE,LO,RE,RO]{} %% clear out all headers
\fancyhead[C]{%
           \begin{flushleft}
          \includegraphics[height=1.5 cm,width=2.5cm]{logo.png}
          \end{flushleft}
          \bgroup
          \def\arraystretch{1.5}% 
          \begin{tabular}{V{4}l|l|l|lV{4}}
            \hlineB{4}
            \multicolumn{4}{V{4}cV{4}}{
                \textbf{\textless{}Document Title\textgreater{}}
            } \\ 
            \hline
            {\footnotesize Type: \textless{}Type\textgreater{}} & 
            {\footnotesize No.: \textless{}Number\textgreater{}} & 
            {\footnotesize Version: \textless{}Document Version\textgreater{}} & 
            {\footnotesize Page: \thepage of \pageref{LastPage}} \\ 
            \hlineB{4}
          \end{tabular}%
          \egroup
}


\begin{document}

\section{Purpose}
This document defines the hardware and software test procedures required for verification of a
self-sealing stem bolt. As we know nothing could be built without bolts. They are a basic component of reverse-ratcheting routing planers.
\section{References}

\subsection{Internal References}

\begin{itemize}
	\item  Self-sealing stem bolt on memory-alpha.fandom.com (\href{https://memory-alpha.fandom.com/wiki/Self-sealing_stem_bolt}{link})
	\item  Self-sealing stem bolt on memory-beta.fandom.com (\href{https://memory-beta.fandom.com/wiki/Self-sealing_stembolt}{link})
\end{itemize}

\section{System Verification}
This document describes the system verification for self-sealing stem bolt. Unfortunately self-sealing stem bolts are mysterious devices of unknown use and origin. The field of application of the self-sealing stem bolts is shrouded in mystery.

\subsection{Objective}

Even though the self-sealing stem bolt is a mysterious device, this document outlines a procedure to determine how one can be tested.

\subsection{Required Equipment}

\begin{itemize}
	\item standard issue Tricorder
	\item 20T NMR spectrometer
	\item quantum combobulator
\end{itemize}



\subsection{Setup and Configuration}

No special setup and configuration is required beyond ensuring calibration of Tricorder, spectrometer and combobulator.

\subsection{Requirements Tested}


\bgroup
\def\arraystretch{1.5}% 
\begin{tabularx}{\textwidth}{|X|}
	\hline
	\textbf{Requirements from Document Number:} \\ Reference the document where the requirements are coming from here \\ \hline
	S3.1.1, S3.2.2, S3.2.3, S3.2.4              \\ \hline
\end{tabularx}
\egroup

\subsubsection{Procedure and Test Worksheet}

\bgroup
\def\arraystretch{1.5}% 
\begin{tabularx}{\textwidth}{|>{\hsize=.8\hsize\linewidth=\hsize}X|>{\hsize=1.2\hsize\linewidth=\hsize}X|}
	\hline
	\textbf{Stardate Test Performed}           & \\ \hline
	\textbf{Starbase Test Performed}           & \\ \hline
	\textbf{Test System Serial \#}             & \\ \hline
	\textbf{Test Computer Serial \#}           & \\ \hline
	\textbf{Functional self-sealing stem bolt} & \\ \hline
	\textbf{Defective self-sealing stem bolt}  & \\ \hline
\end{tabularx}
\egroup




\bgroup

\small
\def\arraystretch{1.5}% 
\begin{xltabular}{\textwidth}{
	|>{\hsize=0.25\hsize\linewidth=\hsize}X
	|>{\hsize=1.85\hsize\linewidth=\hsize}X
	|>{\hsize=1.85\hsize\linewidth=\hsize}X
	|>{\hsize=0.25\hsize\linewidth=\hsize}X
	|>{\hsize=0.8\hsize\linewidth=\hsize}X
	|
	}
	\hline
	\rowcolor{gray}


	Step  &
	Action  &
	Verification  &
	P/F  &
	Result, Notes   \\ \hline \endhead
	\addtocounter{teststepctr}{1} \theteststepctr \ &
	Ensure indicator lights are working by pressing and holding down the on button for 10 seconds.&
	All indicator lights (red, gree, blue) flash first one after another twice and then all together 3 times.
	&

	&

	\\ \hline

	\addtocounter{teststepctr}{1} \theteststepctr \ &
	Ensure correct composition of self-sealing stem bolt by inserting it into a spectrometer &
	The stem bolt should consist of 80\% mixed duranium, aluminum, and steel alloys, 11\% electrically modulated ceramic, and 9\% thermally stabilized plastic.
	&

	&

	\\ \hline



\end{xltabular}
\egroup


\noindent\bgroup
\def\arraystretch{1.5}% 
\begin{tabularx}{\textwidth}{|X|}
	\hline
	\textbf{Summary Report Results:} \\  \hline
	Additional Comments:             \\ \rule{0pt}{12ex}     \\ \hline


	\bgroup
	\def\arraystretch{1.1}% 
	\begin{tabular}{@{}p{0.98in}p{3.2in}p{0.3in}p{1.4in}@{}}
		\rule{0pt}{3ex}                                                  \\
		Completed by: & \hrulefill                  & Date: & \hrulefill \\
		              & Lieutenant junior grade Nog &       &            \\
		              & Junior Engineer             &       &            \\
	\end{tabular}
	\egroup
	\\ \hline


	\bgroup
	\def\arraystretch{1.1}% 
	\begin{tabular}{@{}p{0.98in}p{1.45in}p{1.45in}p{1.6in}@{}}
		\rule{0pt}{.5ex}                                 \\ Result: &
		Pass $\square$ &
		Fail $\square$ &
		Pass with limitations $\square$ \rule{0pt}{.5ex} \\
	\end{tabular}
	\egroup
	\\ \hline



	\bgroup
	\def\arraystretch{1.1}% 
	\begin{tabular}{@{}p{0.98in}p{3.2in}p{0.3in}p{1.4in}@{}}
		\rule{0pt}{3ex}                                                          \\
		Approved By: & \hrulefill                           & Date: & \hrulefill \\
		             & Lieutenant commander Geordi La Forge &       &            \\
		             & Chief Engineer                       &       &            \\
	\end{tabular}
	\egroup
	\\ \hline
\end{tabularx}
\egroup


\section{Document Revision History}



\bgroup

\small
\def\arraystretch{1.5}% 
\begin{xltabular}{\textwidth}{
	|>{\hsize=0.15\hsize\linewidth=\hsize}X
	|>{\hsize=1.7\hsize\linewidth=\hsize}X
	|>{\hsize=0.15\hsize\linewidth=\hsize}X
	|
	}
	\hline
	\rowcolor{gray}


	Ver  &
	Change Description  &
	Author  \\ \hline \endhead

	01 & Release version 1 of self-sealing stem bolt & Nog \\ \hline

\end{xltabular}
\egroup

\end{document}
